%%

\documentclass[final]{siamltex}
\usepackage{epsfig}
\usepackage{geometry}
\usepackage{color}
\usepackage{amsmath}

% definitions used by included articles, reproduced here for
% educational benefit, and to minimize alterations needed to be made
% in developing this sample file.

\newcommand{\pe}{\psi}
\def\d{\delta}
\def\ds{\displaystyle}
\def\e{{\epsilon}}
\def\eb{\bar{\eta}}
\def\enorm#1{\|#1\|_2}
\def\Fp{F^\prime}
\def\fishpack{{FISHPACK}}
\def\fortran{{FORTRAN}}
\def\gmres{{GMRES}}
\def\gmresm{{\rm GMRES($m$)}}
\def\Kc{{\cal K}}
\def\norm#1{\|#1\|}
\def\wb{{\bar w}}
\def\zb{{\bar z}}

% some definitions of bold math italics to make typing easier.
% They are used in the corollary.

\def\bfE{\mbox{\boldmath$E$}}
\def\bfG{\mbox{\boldmath$G$}}

% My commands and macros (comment out if you don't want them)
\def \vec#1{{\bf{#1}}}
\def\pd#1#2{\dfrac{\partial #1}{\partial #2}}
\def \mat#1{\underline{\underline{#1}}}
\def \mylim#1#2{\lim_{#1 \rightarrow #2}}
\newcommand{\bi}{\begin{itemize}}
\newcommand{\ei}{\end{itemize}}
\newcommand{\func}{\mathcal{F}}
\newtheorem{thm}{Theorem}
\newtheorem{cor}[thm]{Corollary}
\newtheorem{lem}[thm]{Lemma}

\title{Your Title\thanks{Submitted for Math 250 Final Project, April 25, 2015}}

% The thanks line in the title should be filled in if there is
% any support acknowledgement for the overall work to be included
% This \thanks is also used for the received by date info, but
% authors are not expected to provide this.

\author{I. M. Author
\thanks{Department of Mathematics, 503 Boston Ave, Tufts University,
Medford, MA 02155. {\it email:} {\tt
email@tufts.edu}  } }


\begin{document}

\maketitle

%%%%%% ABSTRACT %%%%%%%%%%%%%%%%%%%%%%%%%%%%%%%%%%%%%%%%%%%%%%%%
\begin{abstract}
This is an abstract and it should summarize your findings.  Try not to use citations, unless really necessary.   \end{abstract}
%%%%%%%%%%%%%%%%%%%%%%%%%%%%%%%%%%%%%%%%%%%%%%%%%%%%%%%%%%%%

\begin{keywords}
nonlinear PDEs, ??
\end{keywords}

\section*{Acknowledgments}
Did anyone help??

\pagestyle{myheadings}
\thispagestyle{plain}
\markboth{\sc LastName}{\sc SHORT TITLE}

%%%%%%%%%%%%%%%%%%%%%%%%%%%%%%%%%%%%%%%%%%%%%%%%%%%%%%%%%
%% Intro %%
%%%%%%%%%%%%%%%%%%%%%%%%%%%%%%%%%%%%%%%%%%%%%%%%%%%%%%%%%
\section{Introduction} Your paper should have an intro where you give the background information on the PDE and why it is interesting.  This should be where you give
a literature review.  Citations are done like this, \cite{1986BrandtA-aa,2001TrottenbergU_OosterleeC_SchullerA-aa}.

%%%%%%%%%% MHD Equations &&&&&&&&&&&&&&&&&&&&&&&&&&&&&&&&&&&&&&&&&&&&
\section{More Sections}

This is how you do a table.  
\begin{table}[h!]
\centering
{\small
\begin{tabular}{|c|c|c|c|c|c|c|}

\hline

Time&Grid&DOF&Nwt Steps&Avg V-cycles&Work Units&Avg WU/Timestep\\

\hline

1-82&6&595&1&4&0.011&\\

1-82&5&2,079&2&11.&0.118&\\

1-82&4&7,735&2&14.1&0.650&\\

1-82&3&29,799&2&17.8&3.421&$\approx 35$ WU\\

1-82&2&116,935&1&15.7&12.677&\\

1-82&1&463,239&1&5.7&18.358&\\
\hline

83-200&6&595&2&8&0.021&\\

83-200&5&2,079&2&11&0.118&\\

83-200&4&7,735&2&18.4&0.852&\\

83-200&3&29,799&2&21&4.063& $\approx 77$ WU\\

83-200&2&116,935&1&20.9&16.869&\\

83-200&1&463,239&1&16.9&55.036&\\

\hline

\end{tabular}
}

\caption{Number of degrees of freedom (DOF), Newton steps, and V-cycles used at each level and timestep.  The number of work units (WU) or equivalent fine-grid relaxations are also computed here.}
\label{tearingunif}
\end{table}



Look I referenced Table \ref{tearingunif}. 

This is how we itemize.
\begin{itemize}
\item[1.]	I one the sandbox.
\item[2.] I two the sandbox. 
\item[3.]	I three the sandbox.
\item[4.] I four the sandbox. 
\item[5.]	I five the sandbox.
\item[6.] I six the sandbox. 
\item[7.]	I seven the sandbox.
\item[8.] I eight the sandbox. 
\item[9.] That's gross...
\end{itemize}
	
\section{Another Section}
\subsection{With Subsections}

This is how to include a figure.  By the way:
%comments work like this...
\begin{figure}[h!]
\centering
%\includegraphics[scale=0.35]{filename}
\caption{I am a caption.  I really wish I had a figure...}
\label{myfig}
\end{figure}

This is how your reference a figure, like Figure \ref{myfig}.

%%%%%%%%%%%%%%%%%%%%%%%%%%%%%%%%%%%%%%%%%%%%%%%%%%%%%%%%%
%% Conclusion %%
%%%%%%%%%%%%%%%%%%%%%%%%%%%%%%%%%%%%%%%%%%%%%%%%%%%%%%%%%
\section{Discussion}
You should definitely have a conclusion or discussion section.  Summarize your findings and discuss what future aspects may be studied on the topic. 

Oh yes, and don't forget to spellcheck!!!!!!
 
%%%%%%%%%%%%%%%%%%%%%%%%%%%%%%%%%%%%%%%%%%%%%%%%%%%%%%%%%
%% References %%
%%%%%%%%%%%%%%%%%%%%%%%%%%%%%%%%%%%%%%%%%%%%%%%%%%%%%%%%%
\bibliographystyle{plain}	% or "siam", or "alpha", or "abbrv"
				% see other styles (.bst files) in
				% $TEXHOME/texmf/bibtex/bst

\nocite{*}		% list all refs in database, cited or not.

\bibliography{refs}		% bib database file refs.bib

\end{document}

